%This is a very important section, you explain to us how you made it work.
%
%\subsection{Theoretical Background}
%If you are using theoretical concepts, explain them first in this subsection.
%Even if they come from the course (eg. lattices), try to explain the essential
%points \emph{in your own words}. Cite any reference work you used like this
%\cite{TigerBook}. This should convince us that you know the theory behind what
%you coded. 
%
%\subsection{Implementation Details}
%Describe all non-obvious tricks you used. Tell us what you thought was hard and
%why. If it took you time to figure out the solution to a problem, it probably
%means it wasn't easy and you should definitely describe the solution in details
%here. If you used what you think is a cool algorithm for some problem, tell us.
%Do not however spend time describing trivial things (we what a tree traversal
%is, for instance). 
%
%After reading this section, we should be convinced that you knew what you were
%doing when you wrote your extension, and that you put some extra consideration
%for the harder parts.

\subsection{Theoretical Background}
TVM was written with mainly with \cite{Spec} and with help at first of
\cite{Wiki}.

The idea was to implement it with a stack of frame, each frame containing the
currently defined variables and the stack of the function.

At every \emph{invokevirtual}, we push a new frame on the frame stack, setup the
variables in the registers and run the method's byte code. At every return, we
get the return value and pop the frame.

\subsection{Implementation Details}
As there is no dependency, is should easily work on any compliant computer.
If you have \emph{tup} installed (http://gittup.org/tup/), do
`cd tvm \&\& tup` and then the bin is at `build-release/tvm`

Otherwise, the next command should do the trick:
`cd tvm \&\& ./configure \&\& make \&\& make install` and then the bin is at
`./tvm`

\begin{itemize}
	\item to avoid too much IO, we only load classes once by run, so we have
		a manager class keeping loaded instances. It is a singleton by
		the way
	\item we use everywhere shared\_ptr (and unique\_ptr depending of the
		possibility of duplicating or not the elements) to provide a
		simple way of garbage collecting, thus if we have many many
		instances of the same class created, every time the element is
		not needed anymore, it will be deleted (reference counting is
		this case) (see program/fullObject.tool for an example and watch
		the system memory not blowing up)

	\item the specialized classes, like \emph{String}, \emph{System.out} or
		\emph{StringBuilder}, behave just like normal classes but
		redefine the \emph{run\_func()}

	\item every type checking is made via \emph{dynamic\_pointer\_cast}
		(and it's utility function \emph{util::dpc()}, to throw is
		anything goes wrong), there is no formal verification at load
		time

	\item the current state allow running every Tool programs from the
		archive provided, it could easily extended to add new opcode,
		should the need arise (it was written this way, tests driven
		development with the examples programs)

	\item more or less every classes involved in parsing the byte code, have
		a \emph{parse()} function, to return the correct instance of the
		wanted class, as there is no way to return anything more than
		the requested type of the instance in C++

	\item there is a heavy use of macros, which are slowly progressing
		toward templates, as these are type safe but harder to implement
		nicely. You can see some examples in \emph{opcode.cpp} and the
		parsing part is more or less done. There is also some template
		specialization to ease in general with readability

	\item there is some empty parser code, like \emph{interface} existing
		solely to skip part of the file. It is still needed however to
		allow easy extension should it be needed

	\item some opcode, namely \emph{invokespecial} are just implemented as
		pop, as there is no real creation of object, everything is done
		during \emph{new}. There is also the \emph{getstatic} which push
		\emph{SystemOut} directly on the stack, without any verification
		as this is the only one using this opcode (in Tool)
\end{itemize}

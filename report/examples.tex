%Give code examples where your extension is useful, and describe how they work
%with it. Make sure you include examples where the most intricate features of
%your extension are used, so that we have an immediate understanding of what the
%challenges are.
%
%You can pretty-print tool code like this:
%\begin{lstlisting}
%object {
%  def main() : Unit = { println(new A().foo(-41)); }
%}
%
%class A {
%  def foo(i : Int) : Int = {
%    var j : Int;
%    if(i < 0) { j = 0 - i; } else { j = i; }
%    return j + 1;
%  }
%}
%\end{lstlisting}
%
%This section should convince us that you understand how your extension can be
%useful and that you thought about the corner cases.

If we take the following Tool code, it will generate with the reference compiler
the next bytecode (simplified).

\begin{lstlisting}
object MainObject {
	def main(): Unit = {
		println(new A().go());
	}
}

class A {
	def go(): String = {
		var string: String;
		string = "Gone!";
		return string;
	}
}
\end{lstlisting}

object MainObject: CP
\newline
\begin{tabular}{r l l} \hline
	\#14	& Utf8		& java/lang/System	\\
	\#15	& Class		& \#14			\\
	\#16	& Utf8		& out			\\
	\#17	& Utf8		& Ljava/io/PrintStream;	\\
	\#18	& NameAndType	& \#16:\#17		\\
	\#19	& Fieldref	& \#15.\#18		\\
	\#20	& Utf8		& A			\\
	\#21	& Class		& \#20			\\
	\#22	& Methodref	& \#21.\#10		\\
	\#23	& Utf8		& go			\\
	\#24	& Utf8		& ()Ljava/lang/String;	\\
	\#25	& NameAndType	& \#23:\#24		\\
	\#26	& Methodref	& \#21.\#25		\\
	\#27	& Utf8		& java/io/PrintStream	\\
	\#28	& Class		& \#27			\\
	\#29	& Utf8		& println		\\
	\#30	& Utf8		& (Ljava/lang/String;)V	\\
	\#31	& NameAndType	& \#29:\#30		\\
	\#32	& Methodref	& \#28.\#31		\\
\end{tabular}
\newline

object MainObject: main()
\newline
\begin{tabular}{r l l} \hline
	0	& getstatic	& \#19	\\
	3	& new		& \#21	\\
	6	& dup		&	\\
	7	& invokespecial	& \#22	\\
	10	& invokevirtual	& \#26	\\
	13	& invokevirtual	& \#32	\\
	16	& return	&	\\
\end{tabular}
\newline

class A: CP
\newline
\begin{tabular}{r l l} \hline
	\#15	& String	& \#14	\\
\end{tabular}
\newline

class A: go()
\newline
\begin{tabular}{r l l} \hline
	0	& ldc		& \#15	\\
	2	& astore\_1	&	\\
	3	& aload\_1	&	\\
	4	& areturn	&	\\
\end{tabular}
\newline

First we load \emph{MainObject.class}, parse to get the CP, load the bytecode
for every methods and run the asked one (if it is the main object, then, default
to \emph{main()}). As you can see, there is number of back references, so there
is a number of back resolution to do, luckily, we can do it in a single pass, as
there is not any foward one. In our case, some are already resolved during
the parsing stage (as the String ones which do not referee to the UTF8 one).
Next, we parse every methods and its attributes, to finally find the opcodes in
the "Code" attribute, which we run.

Encountering the \emph{new} opcode, we load the new class referenced by it,
namely \emph{A.class}. Here the same happen parsing phase comes in place (if
there was a more complex class hierarchy, it will also load parent class).
Then, with the line 10, we call the function \emph{go()} in the newly loaded
class. A new frame is pushed, and it goes on until it encounter a return, where
it pop the frame and continue running the previous one.
